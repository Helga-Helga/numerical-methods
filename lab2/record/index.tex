\input{../common/head_ru.tex}
\def\passYear{2017}
\def\faculty{физико-технический институт}
\def\department{Кафедра информационной безопасности}
\def\departmentHead{Н. В. Грайворонский}
\def\kind{Дипломна робота}
\def\level{магістр}
\def\specialityCode{8.04030101}
\def\specialityTitle{Прикладная математика}
\def\theme{Решение нелинейных уравнений}
\def\gender{female}
\def\mentorGender{male}
\def\course{3}
\def\group{ФИ-41}
\def\name{Лавягина Ольга Алексеевна}
\def\mentorRank{}
\def\mentorName{Стёпочкина Ирина Валерьевна}
\def\reviewerRank{Rank}
\def\reviewerName{Name}
\def\subject{Методы вычислений}



\begin{document}

\import{1_title/}{title.tex}

\clearpage

\pagenumbering{gobble}
%\import{3_abstract/}{main.tex}

\pagestyle{empty}
\thispagestyle{empty}
\tableofcontents

\clearpage
\pagenumbering{arabic}
\pagestyle{fancy}
\setcounter{page}{2}

\clearpage

\chapter{Исходная система}

Рассматирваем систему вида $Ax = b$, где $A \left( n \times n \right) $ --- матрица системы, $b$ ---
вектор правой части, $x$ --- вектор решения.

Для варианта 8 матрица системы имеет вид
\begin{equation*}
A =
\begin{bmatrix}
    3.81 & 0.25 & 1.28 & 1.75 \\
    2.25 & 1.32 & 5.58 & 0.49 \\
    5.31 & 7.28 & 0.98 & 1.04 \\
    10.39 & 2.45 & 3.35 & 2.28, \\
\end{bmatrix}
\end{equation*}
а вектор правой части ---
\begin{equation*}
b =
\begin{bmatrix}
    4.21 \\
    7.47 \\
    2.38 \\
    11.48 \\
\end{bmatrix}
\end{equation*}

Матрица системы несимметрична, поэтому будем использовать метод Гаусса.

\chapter{Результаты по шагам приведения к треугольной форме матрицы}

Прямой ход --- приведение матрицы к треугольной форме.
Он состоит в следующем:
\begin{enumerate}
  \item нахождение максимального по модулю элемента в матрице
  $$a_{main} = \max_{i, j} \left| a_{ij} \right|, \,
    i, j = 1 \dotsc n;$$
  \item обнуление элементов матрицы в стоблце с максимальным элементом;
  \item переход к пункту 1 с уменьшением размера матрицы
  (выбрасыванимем стоблца и строки с главным элементом), пока не останется матрица $1 \times 1$.
\end{enumerate}

Результат приведения матрицы к диагональному виду по шагам
(строки и столбцы с максимальными элементами не удалялись, так как были использованы в дальшейшем,
но не учитывались в следующих итерациях)

\lstset{inputencoding=utf8, extendedchars=\true}
\lstinputlisting[language=bash,
                 basicstyle=\ttfamily\scriptsize]{../code/get_diagonal}

\chapter{Конечный результат (решение уравнения)}

После того, как была получена треугольная матрица, был применен обратный метод Гаусса.
Составлена система с треугольной матрицей и найдено решение с шестью значимыми цыфрами

\lstset{inputencoding=utf8, extendedchars=\true}
\lstinputlisting[language=bash,
                 basicstyle=\ttfamily\scriptsize]{../code/solution}

\chapter{Вектор невязки}

Для проверки был найден вектор невязки по формуле $r = b - Ax$, где $A$ --- исходная матрица, $x$ ---
наденное решение, $b$ --- столбец правой части

\lstset{inputencoding=utf8, extendedchars=\true}
\lstinputlisting[language=bash,
                 basicstyle=\ttfamily\scriptsize]{../code/residuals}

\chapter{Листинг программы}

Листинг файла main.py
\lstset{inputencoding=utf8, extendedchars=\true}
\lstinputlisting[language=python,
                 basicstyle=\ttfamily\scriptsize]{../code/main.py}

Листинг файла solve.py
\lstset{inputencoding=utf8, extendedchars=\true}
\lstinputlisting[language=python,
                basicstyle=\ttfamily\scriptsize]{../code/solve.py}

\chapter*{Выводы}
\addcontentsline{toc}{chapter}{Выводы}

Система линейных алгебраических уравенений была решена с помощью метода Гаусса.
Была использована схема с выбором главного элемента.

Получен вектор невязки, который показывает погрешность найденного решения.
Полученная погрешность не большая, поэтому решение было найдено достаточно точным.

\end{document}
