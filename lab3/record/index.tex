\documentclass[a4paper,14pt,russian]{extreport}
\usepackage[russian]{babel}

\usepackage{../common/dsturep_ru} % оформление по ДСТУ 3008-95
\usepackage{import}
\usepackage{standalone}
\usepackage{comment}
\usepackage{bbm}

\usepackage{tikz}
\usepackage{tikz-3dplot}
\usetikzlibrary{calc}
\usetikzlibrary{plotmarks}
\usepackage{pgfplots}

%\usepackage{scrextend}
\usepackage{changepage}
\usepackage{caption}
\usepackage{listings}
%\usepackage[title,titletoc]{appendix}
%\usepackage{appendix}
\usepackage{longtable}
%\usepackage{slashbox}
\usepackage{diagbox}
\usepackage{lscape}
\usepackage{algorithmic}
\usepackage{algorithm}

\def\male{male}
\def\female{female}

\bibliographystyle{../common/utf8gost780u}

\usepackage[square,numbers,sort&compress]{natbib}
\renewcommand{\bibnumfmt}[1]{#1.\hfill} % нумерация источников в самом списке — через точку

\def\passYear{2017}
\def\faculty{физико-технический институт}
\def\department{Кафедра информационной безопасности}
\def\departmentHead{Н. В. Грайворонский}
\def\kind{Дипломна робота}
\def\level{магістр}
\def\specialityCode{8.04030101}
\def\specialityTitle{Прикладная математика}
\def\theme{Решение нелинейных уравнений}
\def\gender{female}
\def\mentorGender{male}
\def\course{3}
\def\group{ФИ-41}
\def\name{Лавягина Ольга Алексеевна}
\def\mentorRank{}
\def\mentorName{Стёпочкина Ирина Валерьевна}
\def\reviewerRank{Rank}
\def\reviewerName{Name}
\def\subject{Методы вычислений}



\begin{document}

\import{1_title/}{title.tex}

\clearpage

\pagenumbering{gobble}
%\import{3_abstract/}{main.tex}

\pagestyle{empty}
\thispagestyle{empty}
\tableofcontents

\clearpage
\pagenumbering{arabic}
\pagestyle{fancy}
\setcounter{page}{2}

\clearpage

\chapter{Исходная система}

Рассматирваем систему вида $Ax = b$, где $A \left( n \times n \right) $ --- матрица системы, $b$ ---
вектор правой части, $x$ --- вектор решения.

Для варианта 8 матрица системы имеет вид
\begin{equation*}
A =
\begin{bmatrix}
    3.81 & 0.25 & 1.28 & 1.75 \\
    2.25 & 1.32 & 5.58 & 0.49 \\
    5.31 & 7.28 & 0.98 & 1.04 \\
    10.39 & 2.45 & 3.35 & 2.28 \\
\end{bmatrix},
\end{equation*}
а вектор правой части ---
\begin{equation*}
b =
\begin{bmatrix}
    4.21 \\
    7.47 \\
    2.38 \\
    11.48 \\
\end{bmatrix}.
\end{equation*}

\chapter{Приведение матрицы к диагональному преобладанию}

Нужно привести матрицу к такому виду, чтобы на диагоналях стояли максимальные элементы,
причём сумма не диагональных была меньше самих диагональных элементов.

Для этого поменяем местами первую и четвёртую строки, а также вторую и третью.
\begin{equation*}
A =
\begin{bmatrix}
    10.39 & 2.45 & 3.35 & 2.28 \\
    5.31 & 7.28 & 0.98 & 1.04 \\
    2.25 & 1.32 & 5.58 & 0.49 \\
    3.81 & 0.25 & 1.28 & 1.75 \\
\end{bmatrix},
\end{equation*}
а вектор правой части ---
\begin{equation*}
b =
\begin{bmatrix}
    11.48 \\
    2.38 \\
    7.47 \\
    4.21 \\
\end{bmatrix}.
\end{equation*}

Все строки матрицы имеют нужный вид, кроме второй четвёртой.
От второй строки отнимем четвёртую.
Четвёртую умножим на $-21$ и прибавим к ней первую строку получанной матрицы, умноженную на 8
\begin{equation*}
A =
\begin{bmatrix}
    10.39 & 2.45 & 3.35 & 2.28 \\
    1.5 & 7.03 & -0.3 & -0.71 \\
    2.25 & 1.32 & 5.58 & 0.49 \\
    -3.11 & -14.35 & 0.08 & 18.51 \\
\end{bmatrix},
\end{equation*}
а вектор правой части ---
\begin{equation*}
b =
\begin{bmatrix}
    11.48 \\
    -1.83 \\
    7.47 \\
    -3.43 \\
\end{bmatrix}.
\end{equation*}

От системы $Ax = b$ перейдём к $x = Bx + d$.

Оставляем слева диагональные элементы, вправо переносим все остальные
$$ \begin{cases}
    10.39 x_1 = -2.45 x_2 - 3.35 x_3 - 2.28 x_4 + 11.48, \\
    7.03 x_2 = -1.5 x_1 + 0.3 x_3 + 0.71 x_4 - 1.83, \\
    5.58 x_3 = -2.25 x_1 - 1.32 x_3 - 0.49 x_4 + 7.47, \\
    18.51 x_4 = 3.11 x_1 + 14.35 x_2 - 0.08 x_3 - 3.43.
  \end{cases}$$

Поделим на коэффициенты при диагональных элементах
$$ \begin{cases}
    x_1 = -2.45/10.39 x_2 - 3.35/10.39 x_3 - 2.28/10.39 x_4 + 11.48/10.39, \\
    x_2 = -1.5/7.03 x_1 + 0.3/7.03 x_3 + 0.71/7.03 x_4 - 1.83/7.03, \\
    x_3 = -2.25/5.58 x_1 - 1.32/5.58 x_3 - 0.49/5.58 x_4 + 7.47/5.58, \\
    x_4 = 3.11/18.51 x_1 + 14.35/18.51 x_2 - 0.08/18.51 x_3 - 3.43/18.51.
  \end{cases}$$

Получаем новый вектор
$$b =
\begin{bmatrix}
    11.48/10.39 \\
    -1.83/7.03 \\
    7.47/5.58 \\
    -3.43/18.51 \\
\end{bmatrix}.$$

Формируем матрицу $B$, где на диагонали стоят нули,
а остальные элементы сформированы из правой части новой системы
$$B =
  \begin{bmatrix}
      0 & -2.45/10.39 & -3.35/10.39 & -2.28/10.39 \\
      -1.5/7.03 & 0 & 0.3/7.03 & 0.71/7.03 \\
      -2.25/5.58& -1.32/5.58& 0 & -0.49/5.58 \\
      3.11/18.51 & 14.35/18.51 & -0.08/18.51 & 0 \\
  \end{bmatrix}.$$

\chapter{Результаты первых трёх и последней итерации метода}

\lstset{inputencoding=utf8, extendedchars=\true}
\lstinputlisting[language=bash,
                 basicstyle=\ttfamily\scriptsize]{../code/result}

\chapter{Листинг программы}

Листинг файла \_\_main\_\_.py
\lstset{inputencoding=utf8, extendedchars=\true}
\lstinputlisting[language=python,
                 basicstyle=\ttfamily\scriptsize]{../code/__main__.py}

Листинг файла solve.py
\lstset{inputencoding=utf8, extendedchars=\true}
\lstinputlisting[language=python,
                basicstyle=\ttfamily\scriptsize]{../code/solve.py}

\chapter*{Выводы}
\addcontentsline{toc}{chapter}{Выводы}

Система линейных алгебраических уравенений была решена с помощью метода простой итерации.
Для этого исходная система $Ax = b$ была преобразована так,
что матрица $A$ приобрела диагональное преимущество.

Получен вектор невязки, который показывает погрешность найденного решения.
Полученная точность --- до $10^{-6}$.
Её можно повысить, если задать ошибку $ \varepsilon $ для критерия завершения процесса меньше
(была использована $ \varepsilon = 10^{-4}$).
Исходная система была решена с помощью метода Гаусса.
Полученные решения совпадают.

\end{document}
